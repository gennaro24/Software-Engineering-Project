\documentclass[a4paper,12pt]{article}

% Pacchetti utili
\usepackage{enumitem}
\usepackage{array}
\usepackage{longtable}
\usepackage{float}
\usepackage[utf8]{inputenc}  % Supporto per caratteri UTF-8
\usepackage{graphicx}        % Inserimento immagini
\usepackage{geometry}        % Impostazioni margini
\usepackage{tabularx}
\usepackage{titling}         % Personalizzazione del titolo
\usepackage{fancyhdr}        % Intestazione e piè di pagina
\usepackage[hidelinks]{hyperref}        % Collegamenti nell'indice
\usepackage{lipsum}
\raggedbottom
\newcommand{\req}[4]{
    \paragraph{#1}
    \textbf{Descrizione:} #2 \\
    \textbf{Business Value (BV):} #3 \\
    \textbf{Technical Risk (TR):} #4 \\
}

\geometry{left=3cm, right=3cm, top=3cm, bottom=3cm}

% Personalizzazione intestazione
\pagestyle{fancy}
\fancyhf{}
\fancyfoot[C]{\thepage}

% Logo Università (sostituisci "logo_unisa.png" con il percorso corretto)
\title{
    \includegraphics[width=4cm]{../../resources/logo.png} \\[1cm]  % Logo Università
    \textbf{\Large Università degli Studi di Salerno} \\[0.5cm]
    \textbf{\large Dipartimento di Ingegneria dell'Informazione,\\
    Elettrica e Matematica Applicata} \\[1.5cm]
    {\huge Requirements Engineering}
}

\author{Foschillo Gennaro}
\date{}

\begin{document}

    \maketitle
    \thispagestyle{empty}  % Rimuove il numero di pagina sulla copertina

% Inserire l'indice
    \tableofcontents
    \newpage

% Inizio del documento
    \section{Glossario dei Termini}
% Testo introduttivo sul glossario e sulla descrizione
    In questa sezione vengono forniti i termini utilizzati nel documento e una descrizione generale della categorizzazione dei requisiti. Inoltre, si descrive il simbolismo associato ad ogni categoria di requisiti, che sarà utilizzato successivamente.

    \subsection{Categorizzazione dei Requisiti}
    I requisiti sono suddivisi in due categorie principali: Functional Requirements e Non-Functional Requirements.

    \subsubsection{Functional Requirements (FR)}
    I requisiti funzionali descrivono le funzionalità che l'applicativo deve necessariamente avere,come esso deve reagire a specifici input e cosa il sistema non deve fare.Ogni requisito appartenente a questa categoria verrà indicato con il codice: FR- seguito da un prefisso che indica il sottogruppo di appartenenza e un numero. I requisiti funzionali vengono ulteriormente suddivisi in 3 categorie principali: \textbf{\textit{Individual Functionalities , Data Format and Information , User Interface}}
    \paragraph{Individual Functionalities (IF)}
    Questa categoria descrive le funzionalità offerte dall'applicativo, in termini di: funzionalità di accesso , gestione dell'applicazione e servizi offerti all'utente. Ogni requisito appartenente a questa categoria verrà indicato con il codice: FR-IF seguito da un numero.
    \paragraph{Data Format and Information (DFI)}
    In questo sottogruppo di requisiti viene descritto il formato dei dati gestiti dall'applicazione e il loro mantenimento durante l'utilizzo. Ogni requisito appartenente a questa categoria verrà indicato con il codice: FR-DFI seguito da un numero.
    \paragraph{User Interface (UI)}
% Testo descrittivo su User Interface
    I requisiti con questo prefisso descrivono l'interfaccia utente e la sua usabilità, in termini di:
    \begin{itemize}
        \item \textbf{Dashboard} : Quali informazioni devono essere visualizzate all'interno dell'interfaccia utente.
        \item \textbf{Messaggi di errore} : Quali messaggi di errore devono essere visualizzati a seguito di azioni non permesse dal sistema
        \item \textbf{Usabilità del sistema}
    \end{itemize}
    Ogni requisito appartenente a questa categoria verrà indicato con il codice: FR-UI seguito da un numero.
    \subsubsection{Non-Functional Requirements (NFR)}
% Introduzione alla categoria dei requisiti non funzionali
    I requisiti non funzionali descrivono i vincoli imposti al sistema da sviluppare. Vengono suddivisi nelle seguenti due categorie: \textbf{\textit{Interface to the System , Other Constraints}}.
    Ogni requisito appartenente a questa categoria verrà indicato con il prefisso NFR- seguito dal prefisso del sottogruppo di appartenenza ed un numero.
    \paragraph{Interface to the System (IS)}
    Questo sottogruppo di requisiti non funzionali descrive come il software deve interfacciarsi al sistema in sviluppo, in termini di:
    \begin{itemize}
        \item \textbf{\textit{Strutture dati da utilizzare}}
        \item \textbf{\textit{Tipi di dati}}
        \item \textbf{\textit{Interazione tra componenti }}
        \item \textbf{\textit{Ambiente di sviluppo}}
        \item \textbf{\textit{Framework da utilizzare}}
    \end{itemize}
    \paragraph{Other Constraints (OC)}
    Qui verranno inseriti vincoli generici a cui non è stato possibile attribuire una categorizzazione.
    \subsection{Prioritizzazione dei requisiti}
    In questa sezione verrà descritto il modo in cui vengono assegnate le priorità ai requisiti, considerando due fattori principali: \textbf{\textit{Business Value , Technical Risk}}.
    \subsubsection{Business Value (BV)}
    Indica il grado di necessità di implementazione di un requisito. Viene descritto secondo una scala di 3 valori:
    \begin{itemize}
        \item \textbf{\textit{Must Have (MH)}}: Requisito necessario per il corretto funzionamento del sistema.
        \item  \textbf{\textit{Should Have (SH)}}: Requisito importante ma non necessario per il corretto funzionamento del sistema.
        \item \textbf{\textit{Nice to Have (NH)}} Requisito da soddisfare solo se non comporta un rischio elevato di fallimento.
    \end{itemize}
    \subsubsection{Technical Risk (TR)}
    Indica il rischio di realizzazione di un requisito. Viene misurato attraverso una scala di 3 valori:
    \begin{itemize}
        \item \textbf{\textit{Alto}}: Probabilità di non riuscire a realizzare il requisito nei tempi previsti molto alta
        \item \textbf{\textit{Medio}}: Rischio medio di realizzazione , dovuto a fattori non controllabili
        \item \textbf{\textit{Basso}}: Requisito che si ritiene sicuramente realizzabile nei tempi previsti
    \end{itemize}
    \section{Requirements Definition}
    \subsection{Descrizione}
    \paragraph{} L'obiettivo di questo documento è quello di definire i requisiti di un applicazione per la gestione dei contatti in una rubrica. In particolare, le funzioni principali sono: \textbf{\textit{Aggiunta , Modifica , Cancellazione , Ricerca .}} Di seguito verranno elencati requisiti nel dettaglio
    \subsection{Individual Functionalities (FR-IFxx)}
    \subsubsection{Aggiunta Contatto (FR-IFAx)}
    \begin{table}[h!]
        \centering
        \begin{tabular}{|l|p{8cm}|l|l|}
            \hline
            \textbf{Codice} & \textbf{Descrizione} & \textbf{BV} & \textbf{TR} \\
            \hline
            FR-IFA0 & L'utente può aggiungere un contatto alla rubrica. &
            MH &
            Basso \\
            \hline
            FR-IFA1 & All'aggiunta di un contatto, l'utente può specificare nome e/o cognome & MH & Basso \\
            \hline
            FR-IFA2 & All'aggiunta di un contatto, se non è presente almeno un campo nome o cognome l'aggiunta non viene effettuata. & MH & Basso \\
            \hline
            FR-IFA3 & All'aggiunta di un contatto l'utente può inserire da 0 a 3 indirizzi e-mail. & MH & Medio \\
            \hline
            FR-IFA4 & All'aggiunta di un contatto l'utente può inserire da 0 a 3 numeri di telefono. & MH & Medio \\
            \hline
            FR-IFA5 & All'aggiunta di un contatto , l'utente deve confermare l'aggiunta prima di essere salvato all'interno della lista. & SH & Basso \\
            \hline
            FR-IFA6 & All'aggiunta di un contatto , l'utente può uscire dall'operazione, con conseguente non salvataggio delle modifiche & NH & Medio \\
            \hline
            FR-IFA7 & Alla conferma dell'aggiunta di un contatto andata a buon fine , la lista verrà aggiornata interamente e salvata. & SH & Basso \\
            \hline
        \end{tabular}
        \caption{Tabella dei requisiti per l’aggiunta del contatto (A).}
    \end{table}
    \subsubsection{Rimozione Contatto (FR-IFRx)}
    \begin{table}[H]
        \centering
        \begin{tabular}{|l|p{8cm}|l|l|}
            \hline
            \textbf{Codice} & \textbf{Descrizione} & \textbf{BV} & \textbf{TR} \\
            \hline
            FR-IFR0 & L'utente può rimuovere un contatto presente nella lista dei contatti. & MH & Basso \\
            \hline
            FR-IFR1 & Dopo la rimozione del contatto la lista deve essere aggiornata e salvata. & MH & Basso
            \\
            \hline
            FR-IFR2 & Durante la rimozione l'utente può confermare la decisione. Dopo la conferma, il contatto sarà eliminato dalla lista con conseguente salvataggio e aggiornamento della lista di contatti & SH & Medio
            \\
            \hline
            FR-IFR3 & Durante la rimozione l'utente può uscire dall'operazione, con conseguente non eliminazione del contatto. & NH & Medio
            \\
            \hline
        \end{tabular}
        \caption{Tabella dei requisiti per la rimozione dei contatti (R).}
    \end{table}
    \nopagebreak
    \subsubsection{Modifica Contatto (FR-IFMx)}
    \begin{table}[H]
        \centering
        \begin{tabular}{|l|p{8cm}|l|l|}
            \hline
            \textbf{Codice} & \textbf{Descrizione} & \textbf{BV} & \textbf{TR} \\
            \hline
            FR-IFM0 & L'utente può modificare un contatto solo se presente nella lista dei contatti. & MH & Basso \\
            \hline
            FR-IFM1 & Durante la modifica di un contatto, l'utente può modificare il nome ed il cognome.  & MH & Basso
            \\
            \hline
            FR-IFM2 & Durante la modifica di un contatto , l'utente può eliminare il nome o il cognome , ma non entrambi i campi. & MH & Basso
            \\
            \hline
            FR-IFM3 & Durante la modifica di un contatto, l'utente può modificare, eliminare uno o tutti gli indirizzi e-mail presenti all'interno del contatto.  & MH & Basso
            \\
            \hline
            FR-IFM4 & Durante la modifica di un contatto, l'utente può modificare, eliminare uno o tutti i numeri di telefono presenti all'interno del contatto. & MH & Basso
            \\
            \hline
            FR-IFM5 & Terminata la modifica di un contatto, la lista sarà aggiornata e salvata con le nuove informazioni & MH & Basso
            \\
            \hline
            FR-IFM6 & Terminata la modifica di un contatto , l'utente potrà confermare la scelta, con conseguente salvataggio e aggiornamento della lista dei contatti. & SH & Medio
            \\
            \hline
            FR-IFM7 & Durante la modifica di un contatto , l'utente può uscire dall'operazione, con conseguente salvataggio e aggiornamento della lista dei contatti. & NH & Medio
            \\
            \hline
        \end{tabular}
        \caption{Tabella dei requisiti per la modifica dei contatti (M).}
    \end{table}
    \nopagebreak
    \subsubsection{Ricerca e visualizzazione Contatti (FR-IFSx)}
    \begin{table}[H]
        \centering
        \begin{tabular}{|l|p{8cm}|l|l|}
            \hline
            \textbf{Codice} & \textbf{Descrizione} & \textbf{BV} & \textbf{TR} \\
            \hline
            FR-IFS0 & L'utente può cercare un contatto presente all'interno della lista dei contatti. & MH & Basso \\
            \hline
            FR-IFS1 & L'utente può cercare un contatto inserendo nella ricerca il nome o una sottostringa di esso. & MH & Medio \\
            \hline
            FR-IFS2 & L'utente può cercare un contatto inserendo nella ricerca il cognome o una sottostringa di esso. & MH & Medio \\
            \hline
            FR-IFS3 & L'utente non può cercare un contatto inserendo nella ricerca l'indirizzo e-mail o il numero di telefono. & MH & Basso \\
            \hline
            FR-IFS4 & L'utente, dopo aver performato la ricerca, visualizzerà i risultati in ordine alfabetico se la ricerca è andata a buon fine & MH & Basso \\
            \hline
            FR-IFS5 & L'utente, dopo aver performato la ricerca , non visualizzerà alcun contatto se la ricerca non è andata a buon fine. & MH & Basso \\
            \hline
            FR-IFS6 & L'utente può visualizzare le informazioni di un contatto inserito nella lista. & SH & Basso \\ \hline
        \end{tabular}
        \caption{Tabella dei requisiti per la ricerca e visualizzazione dei contatti (S).}
    \end{table}
    \subsection{Data Format and Information (FR-DFIxx)}
    \begin{table}[H]
        \centering
        \begin{tabular}{|l|p{8cm}|l|l|}
            \hline
            \textbf{Codice} & \textbf{Descrizione} & \textbf{BV} & \textbf{TR} \\
            \hline
            FR-DFI00 & L'applicativo deve salvare i contatti attraverso file, garantendo la persistenza dei dati tra una sessione e l'altra. & MH & Basso \\
            \hline
            FR-DFI01 & L'applicativo deve caricare il file contenente i contatti ad ogni avvio per popolare la rubrica. & MH & Basso \\
            \hline
            FR-DFI02 & All'avvio dell'applicazione, se il file da caricare è vuoto o non esiste, verrà creata una rubrica vuota. & MH & Basso \\
            \hline
            FR-DFI03 & Ogni contatto che viene salvato su file deve avere il seguente formato: nome, cognome (almeno uno dei due) , 0-3 indirizzi e-mail , 0-3 numeri di telefono. & MH & Basso \\
            \hline
            FR-DFI04 & L'applicativo deve consentire il caricamento manuale di una lista di contatti. & MH & Basso \\ \hline
            FR-DFI05 & L'applicativo deve consentire il salvataggio manuale su file dell'intera lista di contatti & MH & Basso \\ \hline
        \end{tabular}
        \caption{Tabella dei requisiti per le informazioni sui dati , formato dei dati e persistenza. (S).}
    \end{table}
    \subsection{User Interface (FR-UIxx)}
    \subsubsection{Dashboard (FR-UIDx)}

    \begin{longtable}{|l|p{8cm}|l|l|}
        \caption{Tabella dei requisiti per la Dashboard.} \\

        \hline
        \textbf{Codice} & \textbf{Descrizione} & \textbf{BV} & \textbf{TR} \\
        \hline
        \endfirsthead

        \multicolumn{4}{c}{\footnotesize \textit{Tabella dei requisiti per la Dashboard (continuo)}}\\
        \hline
        \textbf{Codice} & \textbf{Descrizione} & \textbf{BV} & \textbf{TR} \\
        \hline
        \endhead

        FR-UID0  & L'utente all'avvio dell'applicazione dovrà visualizzare un messaggio
        di caricamento fino a che il file in cui sono presenti i contatti non sarà caricato. & MH & Basso \\
        \hline
        FR-UID1  & Dopo il caricamento, l'utente visualizzerà la Dashboard. Se il file è vuoto,
        l'utente non visualizzerà nessun contatto. & MH & Basso \\
        \hline
        FR-UID2  & All'interno della Dashboard l'utente visualizzerà la lista dei contatti ordinati alfabeticamente.
        & MH & Basso \\
        \hline
        FR-UID3  & All'interno della Dashboard l'utente potrà aggiungere un contatto
        attraverso l'apposito bottone \texttt{+}. & MH & Basso \\
        \hline
        FR-UID4  & All'interno della Dashboard l'utente potrà modificare un contatto
        attraverso l'apposito bottone di modifica. & MH & Basso \\
        \hline
        FR-UID5  & All'interno della Dashboard l'utente potrà scorrere la lista dei contatti. & MH & Basso \\
        \hline
        FR-UID6  & Quando l'utente selezionerà l'opzione \texttt{+} (F-UID3), visualizzerà un menù contestuale. & MH & Basso \\
        \hline
        FR-UID7  & All'interno del menù contestuale, l'utente potrà inserire, attraverso
        due campi di testo separati, nome e cognome. & MH & Basso \\
        \hline
        FR-UID8  & All'interno del menù contestuale, l'utente potrà inserire,
        attraverso un massimo di 3 campi di testo, gli indirizzi e-mail. & MH & Basso \\
        \hline
        FR-UID9  & All'interno del menù contestuale, l'utente potrà inserire,
        attraverso un massimo di 3 campi di testo, i numeri di telefono. & MH & Basso \\
        \hline
        FR-UID10 & All'interno del menù contestuale, l'utente potrà confermare
        l'aggiunta attraverso un campo di conferma. & SH & Medio \\
        \hline
        FR-UID11 & All'interno del menù contestuale, l'utente potrà tornare alla dashboard
        senza effettuare l'aggiunta attraverso un campo \texttt{Annulla}. & NH & Medio \\
        \hline
        FR-UID12 & All'interno della Dashboard, il bottone di modifica sarà disattivato
        se non sono presenti contatti all'interno della lista. & MH & Basso \\
        \hline
        FR-UID13 & Quando l'utente selezionerà l'opzione di modifica contatto, visualizzerà
        un menù contestuale (come descritto dai req. FR-UID7--FR-UID11) in cui potrà
        modificare i campi esistenti, eliminarli o aggiungerli ove possibile. & MH & Basso \\
        \hline
        FR-UID14 & All'interno della Dashboard l'utente potrà eliminare un contatto
        attraverso l'apposito bottone di eliminazione, cliccando sul contatto. & MH & Basso \\
        \hline
        FR-UID15 & Il bottone di eliminazione è utilizzabile solo se la lista di contatti non è vuota. & MH & Basso \\ \hline
        FR-UID16  & Il bottone di modifica è utilizzabile solo se la lista di contatti non è vuota. & MH & Basso \\ \hline
        FR-UID17 & al click del bottone di eliminazione, si aprirà un menù contestuale per confermare o tornare indietro. & SH & Basso \\ \hline
        FR-UID18 & All'interno della Dashboard è presente una barra di ricerca, in cui l'utente può cercare i contatti per nome e/o cognome. L'utente può confermare la ricerca attraverso l'apposito bottone "Cerca" & MH & Basso \\ \hline
        FR-UID19 & Quando la ricerca verrà effettuata, l'utente visualizzerà la lista dei contatti coerente con la ricerca in ordine alfabetico. Se non è presente alcun contatto che corrisponde alla ricerca, l'utente visualizzerà la lista dei contatti vuota. & MH & Basso \\ \hline
        FR-UID20 & Quando la ricerca viene confermata, dopo la visualizzazione dei risultati , l'utente può ritornare alla Dashboard attraverso il bottone "Indietro" & SH & Basso \\ \hline
        FR-UID21 & L'utente può ritornare alla Dashboard in ogni momento premendo ESC & NH & Basso \\ \hline
        FR-UID22 & L'utente può caricare, salvare file di contatti attraverso due bottoni all'interno della Dashboard. & MH & Basso \\ \hline
        FR-UID23 & L'utente può visualizzare le informazioni di un contatto cliccando su di esso & SH & Basso \\ \hline
    \end{longtable}
    \subsubsection{Error Messages (FR-UIEx)}
    \begin{table}[H]
        \centering
        \begin{tabular}{|l|p{8cm}|l|l|}
            \hline
            \textbf{Codice} & \textbf{Descrizione} & \textbf{BV} & \textbf{TR} \\
            \hline
            FR-UIE1 &
            All'interno del contesto di modifica e aggiunta di un contatto,
            se l'utente non inserisce almeno un campo tra \texttt{nome} e \texttt{cognome}
            ,il bottone di conferma verrà disabilitato e  verrà visualizzato il messaggio di errore:
            \textit{“Uno dei campi nome e cognome non può essere vuoto.”} & MH & Basso \\
            \hline

            FR-UIE2 &
            Se l'utente prova ad inserire più di 3 indirizzi e-mail per lo stesso contatto,
            verrà visualizzato un messaggio di errore:
            \textit{“Non puoi inserire più di 3 indirizzi e-mail per contatto.”} & MH & Basso \\
            \hline

            FR-UIE3 &
            Se l'utente prova ad inserire più di 3 numeri di telefono per lo stesso contatto,
            verrà visualizzato un messaggio di errore:
            \textit{“Non puoi inserire più di 3 numeri di telefono per contatto.”} & MH & Basso \\
            \hline

            FR-UIE4 &
            Se l'utente inserisce un indirizzo e-mail in un formato non valido
            (e.g. privo di chiocciola \texttt{@}),alla conferma verrà visualizzato un messaggio di errore:
            \textit{“L’indirizzo e-mail inserito non è valido.”} & SH & Basso \\
            \hline

            FR-UIE5 &
            Se l'utente inserisce un numero di telefono in un formato ritenuto non valido
            (es. lettere al posto di cifre), alla conferma verrà visualizzato un messaggio di errore:
            \textit{“Il numero di telefono inserito non è valido.”} & SH & Basso \\
            \hline

            FR-UIE6 &
            Se l'utente tenta di rimuovere o modificare un contatto che non esiste più in lista
            (ad es. è stato rimosso durante l’operazione da un altro processo o è corrotto),
            verrà visualizzato un messaggio di errore:
            \textit{“Il contatto selezionato non esiste o è già stato eliminato.”} & SH & Basso \\
            \hline
            FR-UIE7 & Se l'utente effettua una ricerca senza risultati, visualizzerà un messaggio il messaggio di errore "Nessun risultato corrispondente ai criteri di ricerca" & SH & Basso. \\ \hline
        \end{tabular}
        \caption{Tabella dei requisiti per la gestione dei messaggi di errore (E).}
    \end{table}
    \subsubsection{Usability (FR-UIUx)}
    \begin{table}[H]
        \centering
        \begin{tabular}{|l|p{8cm}|l|l|}
            \hline
            \textbf{Codice} & \textbf{Descrizione} & \textbf{BV} & \textbf{TR} \\
            \hline
            FR-UIU0 &
            L'interfaccia deve mantenere uno stile coerente in tutte le viste (Dashboard, menù contestuali, ecc.),
            con pulsanti e label sempre nella stessa posizione e dimensione.
            & MH & Basso \\
            \hline
            FR-UIU1 &
            I testi e i bottoni dell'interfaccia devono avere dimensioni tali
            da garantire una leggibilità adeguata su uno schermo di almeno 1280x720.
            & SH & Basso \\
            \hline
            FR-UIU2 &
            Le operazioni principali (Aggiunta, Modifica, Rimozione di un contatto)
            non devono richiedere più di 3 click per essere completate, dall'apertura della Dashboard al salvataggio finale.
            & SH & Medio \\
            \hline
            FR-UIU3 &
            Il sistema deve fornire un feedback visivo chiaro quando l'utente esegue operazioni di ricerca, aggiunta, modifica o rimozione.
            & SH & Basso \\
            \hline
            FR-UIU4 &
            Le icone utilizzate (per aggiungere, modificare, rimuovere un contatto)
            devono essere intuitive e coerenti con gli standard di design comuni (ad es. un'icona “+” per aggiungere).
            & MH & Basso \\
            \hline
            FR-UIU5 &
            In caso di errori (come da sezione FR-UIEx), l'utente deve visualizzare messaggi chiari e comprensibili,
            con eventuali suggerimenti su come risolvere il problema.
            & MH & Basso \\
            \hline
            FR-UIU6 &
            Il sistema deve impedire l’avvio di un’operazione (es. modifica/eliminazione)
            se non è presente alcun contatto selezionato, evitando click “inutili”.
            & SH & Basso \\
            \hline
        \end{tabular}
        \caption{Tabella dei requisiti di usabilità (UIU).}
    \end{table}
    \subsection{Interface to the System (NFR-ISxx)}
    \begin{table}[H]
        \centering
        \begin{tabular}{|l|p{8cm}|l|l|}
            \hline
            \textbf{Codice} & \textbf{Descrizione} & \textbf{BV} & \textbf{TR} \\
            \hline
            NFR-IS00 & L'applicativo deve essere sviluppato \textbf{\textit{Java}} , compreso di interfaccia grafica realizzata con il framework \textbf{\textit{JavaFX}} , utilizzando il pattern MVC. & MH & Basso \\
            \hline
            NFR-IS01 & L'applicativo deve utilizzare file \textbf{\textit{JSON}} per la memorizzazione e caricamento dei dati riguardanti i contatti, seguendo il seguente pattern: \texttt{\{ "contacts": [ \{ "name": "...", "surname": "...", "emails": [...], "phones": [...] \}, ... ]\}}. & MH & Basso \\
            \hline
            NFR-IS02 & Il campo "nome" e "cognome" sono identificati rispettivamente da due stringhe, ognuna di lunghezza massima di 50 caratteri. & MH & Basso \\
            \hline
            NFR-IS03 & Il campo e-mail è identificato da una stringa che presenta il seguente pattern: nomeEmail@mailAgent.dominio . Inoltre, non devono essere presenti caratteri speciali o spazi. & MH & Basso \\
            \hline
            NFR-IS04 & Il campo numero di telefono è identificato da una stringa contenente solo cifre [0-9], con una lunghezza minima di 3 e massima di 15. Non sono ammessi spazi o caratteri speciali. & MH & Basso \\
            \hline
            NFR-IS05 & La struttura dati da utilizzare è L'ObservableList, così che i cambiamenti vengano riflessi in modo automatico all'interno dell'interfaccia grafica. & SH & Basso \\
            \hline
            \hline
        \end{tabular}
        \caption{Tabella dei requisiti di Interfacciamento al Sistema in Sviluppo (IS).}
    \end{table}
    \subsubsection{Other Constraints NFR-OCxx}
    \begin{table}[H]
        \centering
        \begin{tabular}{|l|p{8cm}|l|l|}
            \hline
            \textbf{Codice} & \textbf{Descrizione} & \textbf{BV} & \textbf{TR} \\
            \hline
            NFR-OC01 & L'applicativo deve garantire un tempo di caricamento con 1000 contatti di massimo 3 secondi. & SH & Medio \\
            \hline
            NFR-OC02 & L'applicativo deve evitare crash improvvisi senza un report dell'errore dettagliato. & NH & Medio \\
            \hline
        \end{tabular}
        \caption{Tabella di requisiti non funzionali generici (OC).}
    \end{table}
    \section{Use Cases (UC-xx)}
    \subsection*{Caso d'Uso: [UC-01]}

    \begin{table}[htbp]
        \centering
        \begin{tabularx}{\textwidth}{|p{0.4\textwidth}|X|}
            \hline
            \textbf{Dettagli Caso d'Uso} & \textbf{Flusso di Eventi Normale} \\
            \hline
            \parbox[t]{0.38\textwidth}{%
                \textbf{Nome:} [Aggiunta Contatto] \\[1ex]
                \textbf{Attore:} [Utente] \\[1ex]
                \textbf{Precondizioni:} \newline 1. L'utente avvia l'applicazione. \newline 2. L'utente supera la fase di caricamento e visualizza la Dashboard. \newline 3. L'utente clicca sul pulsante di aggiunta contatto "+". \newline 4. L'applicativo mostra il menù contestuale di aggiunta contatto. \\[1ex]
                \textbf{Postcondizioni:} \newline Il contatto viene aggiunto alla lista dei contatti , che viene salvata.\newline
            }
            &
            \parbox[t]{\linewidth}{%
                \begin{enumerate}[noitemsep, leftmargin=*]
                    \item Step 1: L'utente digita nome e cognome
                    \item Step 2: L'utente inserisce da 0 a 3 e-mail
                    \item Step 3: L'utente inserisce da 0 a 3 numeri di telefono
                    \item Step 4: L'utente conferma l'aggiunta
                \end{enumerate}
            } \\
            \hline
            \multicolumn{2}{|p{\textwidth}|}{%
                \textbf{Flussi di Eventi Alternativi:} \newline \textbf{(1)} \newline Step 1a: nome e cognome inseriti non rispettano i requisiti. \newline Step 1b: La sezione di conferma è disabilitata. \newline Step 1c: ritorna allo Step 1. \newline \textbf{(2)} \newline Step 2a: L'utente inserisce uno o più mail non conformi ai requisiti \newline Step 2b: La conferma produce un messaggio di errore. \newline Step 2c: Ritorna allo Step 2. \newline \textbf{(3)} \newline Step 3a: l'utente inserisce uno o più numeri di telefono non conformi ai requisiti. \newline Step 3b: La conferma produce un messaggio di errore \newline Step 3c: Ritorna allo Step 3
            } \\
            \hline
        \end{tabularx}
        \caption{Aggiunta Contatto}
    \end{table}
    \newpage
    \subsection*{Caso d'Uso: [UC-02]}

    \begin{table}[htbp]
        \centering
        \begin{tabularx}{\textwidth}{|p{0.4\textwidth}|X|}
            \hline
            \textbf{Dettagli Caso d'Uso} & \textbf{Flusso di Eventi Normale} \\
            \hline
            \parbox[t]{0.38\textwidth}{%
                \textbf{Nome:} [Modifica Contatto] \\[1ex]
                \textbf{Attore:} [Utente] \\[1ex]
                \textbf{Precondizioni:} \newline 1. L'utente avvia l'applicazione. \newline 2. L'utente supera la fase di caricamento e visualizza la Dashboard. \newline3. La lista di contatti presenta almeno un contatto.  \newline 4. L'utente clicca sul pulsante di modifica contatto. \newline 5. L'applicativo mostra il menù contestuale di modifica contatto. \\[1ex]
                \textbf{Postcondizioni:} \newline Il contatto viene modificato, la lista dei contatti viene aggiornata e salvata.\newline
            }
            &
            \parbox[t]{\linewidth}{%
                \begin{enumerate}[noitemsep, leftmargin=*]
                    \item Step 1: L'utente modifica nome e cognome
                    \item Step 2: L'utente modifica da 0 a 3 e-mail
                    \item Step 3: L'utente modifica da 0 a 3 numeri di telefono
                    \item Step 4: L'utente conferma la modifica.
                \end{enumerate}
            } \\
            \hline
            \multicolumn{2}{|p{\textwidth}|}{%
                \textbf{Flussi di Eventi Alternativi:} \newline \textbf{(1)} \newline Step 1a: nome e cognome inseriti non rispettano i requisiti. \newline Step 1b: La sezione di conferma è disabilitata. \newline Step 1c: ritorna allo Step 1. \newline \textbf{(2)} \newline Step 2a: L'utente modifica uno o più mail , rendendole non conformi ai requisiti \newline Step 2b: La conferma produce un messaggio di errore. \newline Step 2c: Ritorna allo Step 2. \newline \textbf{(3)} \newline Step 3a: l'utente modifica uno o più numeri di telefono, rendendoli non conformi ai requisiti. \newline Step 3b: La conferma produce un messaggio di errore \newline Step 3c: Ritorna allo Step 3
            } \\
            \hline
        \end{tabularx}
        \caption{Modifica Contatto}
    \end{table}
    \newpage
    \subsection*{Caso d'Uso: [UC-03]}

    \begin{table}[htbp]
        \centering
        \begin{tabularx}{\textwidth}{|p{0.4\textwidth}|X|}
            \hline
            \textbf{Dettagli Caso d'Uso} & \textbf{Flusso di Eventi Normale} \\
            \hline
            \parbox[t]{0.38\textwidth}{%
                \textbf{Nome:} [Rimozione Contatto] \\[1ex]
                \textbf{Attore:} [Utente] \\[1ex]
                \textbf{Precondizioni:} \newline 1. L'utente avvia l'applicazione. \newline 2. L'utente supera la fase di caricamento e visualizza la Dashboard. \newline 3. La lista dei contatti deve contenere almeno un contatto. \newline \\[1ex]
                \textbf{Postcondizioni:} \newline Il contatto viene rimosso dalla lista dei contatti , che viene aggiornata e salvata.\newline
            }
            &
            \parbox[t]{\linewidth}{%
                \begin{enumerate}[noitemsep, leftmargin=*]
                    \item Step 1: L'utente clicca su un contatto
                    \item Step 2: L'utente clicca il pulsante "rimuovi contatto"
                    \item Step 3: L'applicativo mostra il menù contestuale di rimozione
                    \item Step 4: L'utente conferma la rimozione
                \end{enumerate}
            } \\
            \hline
            \multicolumn{2}{|p{\textwidth}|}{%
                \textbf{Flussi di Eventi Alternativi:} \newline \textbf{(1)} \newline Step 3a: L'utente torna indietro. \newline Step 3b: L'utente ritorna alla Dashboard.\newline Step 3c: ritorna allo Step 1. \newline
            } \\
            \hline
        \end{tabularx}
        \caption{Rimozione Contatto}
    \end{table}
    \newpage
    \subsection*{Caso d'Uso: [UC-04]}

    \begin{table}[htbp]
        \centering
        \begin{tabularx}{\textwidth}{|p{0.4\textwidth}|X|}
            \hline
            \textbf{Dettagli Caso d'Uso} & \textbf{Flusso di Eventi Normale} \\
            \hline
            \parbox[t]{0.38\textwidth}{%
                \textbf{Nome:} [Ricerca Contatto] \\[1ex]
                \textbf{Attore:} [Utente] \\[1ex]
                \textbf{Precondizioni:} \newline 1. L'utente avvia l'applicazione. \newline 2. L'utente supera la fase di caricamento e visualizza la Dashboard. \newline 3. La lista dei contatti deve contenere almeno un contatto. \newline \\[1ex]
                \textbf{Postcondizioni:} \newline L'utente visualizza la lista di contatti corrispondente alla ricerca.\newline
            }
            &
            \parbox[t]{\linewidth}{%
                \begin{enumerate}[noitemsep, leftmargin=*]
                    \item Step 1: L'utente clicca sulla barra di ricerca
                    \item Step 2: L'utente digita una stringa o sottostringa del nome e/o del cognome del contatto da cercare.
                    \item Step 3: L'utente conferma cliccando il bottone di ricerca
                \end{enumerate}
            } \\
            \hline
            \multicolumn{2}{|p{\textwidth}|}{%
                \textbf{Flussi di Eventi Alternativi:} \newline \textbf{(1)} \newline Step 2a: L'utente inserisce una stringa che non corrisponde a nessun nome e/o cognome nella lista dei contatti \newline Step 2b: L'utente conferma la ricerca.\newline Step 2c: L'utente visualizza una lista vuota con un messaggio di errore: "Nessun risultato corrispondente ai criteri di ricerca".  \newline Step 2d: ritorna allo Step 1.
            } \\
            \hline
        \end{tabularx}
        \caption{Ricerca Contatto}
    \end{table}
\end{document}


